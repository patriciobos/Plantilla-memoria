\chapter{Diseño e implementación} % Main chapter title

\label{Chapter3} % Change X to a consecutive number; for referencing this chapter elsewhere, use \ref{ChapterX}

\definecolor{mygreen}{rgb}{0,0.6,0}
\definecolor{mygray}{rgb}{0.5,0.5,0.5}
\definecolor{mymauve}{rgb}{0.58,0,0.82}

%%%%%%%%%%%%%%%%%%%%%%%%%%%%%%%%%%%%%%%%%%%%%%%%%%%%%%%%%%%%%%%%%%%%%%%%%%%%%
% parámetros para configurar el formato del código en los entornos lstlisting
%%%%%%%%%%%%%%%%%%%%%%%%%%%%%%%%%%%%%%%%%%%%%%%%%%%%%%%%%%%%%%%%%%%%%%%%%%%%%
\lstset{ %
  backgroundcolor=\color{white},   % choose the background color; you must add \usepackage{color} or \usepackage{xcolor}
  basicstyle=\footnotesize,        % the size of the fonts that are used for the code
  breakatwhitespace=false,         % sets if automatic breaks should only happen at whitespace
  breaklines=true,                 % sets automatic line breaking
  captionpos=b,                    % sets the caption-position to bottom
  commentstyle=\color{mygreen},    % comment style
  deletekeywords={...},            % if you want to delete keywords from the given language
  %escapeinside={\%*}{*)},          % if you want to add LaTeX within your code
  %extendedchars=true,              % lets you use non-ASCII characters; for 8-bits encodings only, does not work with UTF-8
  %frame=single,	                % adds a frame around the code
  keepspaces=true,                 % keeps spaces in text, useful for keeping indentation of code (possibly needs columns=flexible)
  keywordstyle=\color{blue},       % keyword style
  language=[ANSI]C,                % the language of the code
  %otherkeywords={*,...},           % if you want to add more keywords to the set
  numbers=left,                    % where to put the line-numbers; possible values are (none, left, right)
  numbersep=5pt,                   % how far the line-numbers are from the code
  numberstyle=\tiny\color{mygray}, % the style that is used for the line-numbers
  rulecolor=\color{black},         % if not set, the frame-color may be changed on line-breaks within not-black text (e.g. comments (green here))
  showspaces=false,                % show spaces everywhere adding particular underscores; it overrides 'showstringspaces'
  showstringspaces=false,          % underline spaces within strings only
  showtabs=false,                  % show tabs within strings adding particular underscores
  stepnumber=1,                    % the step between two line-numbers. If it's 1, each line will be numbered
  stringstyle=\color{mymauve},     % string literal style
  tabsize=2,	                   % sets default tabsize to 2 spaces
  title=\lstname,                  % show the filename of files included with \lstinputlisting; also try caption instead of title
  morecomment=[s]{/*}{*/}
}


%----------------------------------------------------------------------------------------
%	SECTION 1
%----------------------------------------------------------------------------------------
\section{Detalle del hardware por etapas}
\subsection{Circuito de selección de modo}
Considerando el caso límite donde el nivel del acumulador es bajo y el MC ordena al relay (RL) con retención cambiar su estado, la energía necesaria para excitar al RL puede generar una caída de tensión significativa en el acumulador y resultar en un circuito totalmente desenergizado sin posibilidad de volver a entrar en operación a no ser mediante la intervención humana.\\
Un caso extremo como el mencionado, implica que el HW posea un mecanismo para restablecer su operación tras largos periodos de tiempo aun después de haberse vaciado el acumulador.\\
En la figura \ref{fig:ctoselecciondemodo} se implementó el RL sin retención presentado en \ref{fig:relay} y su circuito de mando. Los terminales del TI, se encuentran conectados por defecto a la etapa de conversión y acumulacion de energía (terminales \textit{NC1} y \textit{NC2}). Una vez que la tensión en bornes del acumulador alcanza el valor de mínimo para que el conversor DC/DC entre en operación, también la electrónica sin la necesidad de una intervención externa.\\
% TODO: \usepackage{graphicx} required
\begin{figure}[h]
	\centering
	\includegraphics[width=0.7\linewidth]{Figures/cto_seleccion_de_modo}
	\caption{circuito selector de modo basado en el relay HF115-005-2ZS4}
	\label{fig:ctoselecciondemodo}
\end{figure}\\
Cada vez que se desea tomar una medición de corriente, el MC enciende la bobina del relay y conecta el TI al circuito de medición de valor RMS. Al \\

\subsection{Puente rectificador de onda completa}
Para el prototipo desarrollado en este trabajo, se replicó el circuito rectificador de onda completa presentado en la FIGURA TESIS YILMAZ. Con el objeto de reducir el espacio ocupado en el PCB, se optó por utilizar un chip DMHC3025 en encapsulado SOIC-8 [REFERENCIA A DATASHEET].\\
El encapsulado de la \ref{fig:esquematicointernorctificadormosfet} alberga un puente H formado dos transistores MOSFET tipo P y dos tipo N. Las características eléctricas de relevancia para esta aplicacion son presentadas en la tabla \ref{tabla_transistores_rectificador}.
\begin{table}[h]
	\centering
	\caption{Características eléctricas de los transistores que forman parte del rectificador de onda completa.}
	\begin{tabular}{cc} 
		\hline
		\textbf{\textbf{Parámetro}} & \begin{tabular}[c]{@{}c@{}}\textbf{}\\\textbf{\textbf{Valor máximo}}\end{tabular}  \\ 
		\hline
		Rds                         & 25 miliohms                                                                        \\
		Id                          & 6 A                                                                                \\
		Vdss                        & 30 V                                                                               \\
		\hline
	\end{tabular}\label{tabla_transistores_rectificador}
\end{table}

% TODO: \usepackage{graphicx} required
\begin{figure}[h]
	\centering
	\includegraphics[width=0.7\linewidth]{Figures/esquematico_interno_rctificador_mosfet}
	\caption{Pinout y esquematico interno del DMHC3025 [referencia al datasheet].}
	\label{fig:esquematicointernorctificadormosfet}
\end{figure}
El circuito final implementado encargado de la recitificacion y filtrado de la tension DC a la salida se presenta en la figura \ref{fig:ctorectificacionfiltrado}. Dos diodos zener \textit{D1} y \textit{D2} en antiparalelo protegen el a los transistores  de picos de sobretensión recortando la onda a 27 V para cada semiciclo.\\
% TODO: \usepackage{graphicx} required
\begin{figure}
	\centering
	\includegraphics[width=0.7\linewidth]{Figures/cto_rectificacion_filtrado}
	\caption{Etapa de rectificacion, filtrado y acumulacion implementada.}
	\label{fig:ctorectificacionfiltrado}
\end{figure}\\
A su salida, un conjunto de capacitores cerámicos \textit{C5}, \textit{C6} y \textit{C7} filtran ruido de alta frecuencia para luego acumular toda la energía en el SC conectado a J5.\\

 
\section{Firmware implementado}

\section{Servicios de Backend}

\section{Integraci\'{o}n de la red LoRaWAN}

\section{Base de Datos}

\section{GUI basada en Grafana}

