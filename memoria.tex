%%%%%%%%%%%%%%%%%%%%%%%%%%%%%%%%%%%%%%%%%
% Masters/Doctoral Thesis 
% LaTeX Template
% Version 2.3 (25/3/16)
%
% This template has been downloaded from:
% http://www.LaTeXTemplates.com
%
% Version 2.x major modifications by:
% Vel (vel@latextemplates.com)
%
% This template is based on a template by:
% Steve Gunn (http://users.ecs.soton.ac.uk/srg/softwaretools/document/templates/)
% Sunil Patel (http://www.sunilpatel.co.uk/thesis-template/)
%
% Template license:
% CC BY-NC-SA 3.0 (http://creativecommons.org/licenses/by-nc-sa/3.0/)
%
%%%%%%%%%%%%%%%%%%%%%%%%%%%%%%%%%%%%%%%%%

%----------------------------------------------------------------------------------------
%	PACKAGES AND OTHER DOCUMENT CONFIGURATIONS
%----------------------------------------------------------------------------------------

\documentclass[
11pt, % The default document font size, options: 10pt, 11pt, 12pt
%oneside, % Two side (alternating margins) for binding by default, uncomment to switch to one side
%chapterinoneline,% Have the chapter title next to the number in one single line
%english, % ngerman for German
spanish,
singlespacing, % Single line spacing, alternatives: onehalfspacing or doublespacing
%draft, % Uncomment to enable draft mode (no pictures, no links, overfull hboxes indicated)
%nolistspacing, % If the document is onehalfspacing or doublespacing, uncomment this to set spacing in lists to single
%liststotoc, % Uncomment to add the list of figures/tables/etc to the table of contents
%toctotoc, % Uncomment to add the main table of contents to the table of contents
parskip, % Uncomment to add space between paragraphs
%nohyperref, % Uncomment to not load the hyperref package
headsepline, % Uncomment to get a line under the header
]{MastersDoctoralThesis} % The class file specifying the document structure



\usepackage[utf8]{inputenc} % Required for inputting international characters
\usepackage[T1]{fontenc} % Output font encoding for international characters

\usepackage{palatino} % Use the Palatino font by default

\usepackage[backend=bibtex,natbib=true, style=numeric, sorting=none]{biblatex} % Use the bibtex backend for bibliography

\addbibresource{references.bib} % The filename of the bibliography

\usepackage[autostyle=true]{csquotes} % Required to generate language-dependent quotes in the bibliography

\usepackage{caption}
\usepackage{subcaption}

%------------------------
\usepackage{listings}

%\usepackage[hyphens]{url}
%\usepackage[hidelinks]{hyperref}
%\hypersetup{breaklinks=true}
\urlstyle{same}
%\usepackage{cite}
\hyphenation{biblatex}
%--------------------------

\usepackage{color}

%
%----------------------------------------------------------------------------------------
%	MARGIN SETTINGS
%----------------------------------------------------------------------------------------

\geometry{
	paper=a4paper, % Change to letterpaper for US letter
	inner=2cm, % Inner margin
	outer=3.3cm, % Outer margin
	bindingoffset=2cm, % Binding offset
	top=1.5cm, % Top margin
	bottom=1.5cm, % Bottom margin
	%showframe,% show how the type block is set on the page
}

%----------------------------------------------------------------------------------------
%	INFORMACIÓN DE LA MEMORIA
%----------------------------------------------------------------------------------------

\thesistitle{Título del trabajo} % El títulos de la memoria, se usa en la carátula y se puede usar el cualquier lugar del documento con el comando \ttitle
\degree{Especialista en Sistemas Embebidos } % Nombre del grado, se usa en la carátula y se puede usar el cualquier lugar del documento con el comando \degreename
\author{Nombre del Autor} % Tu nombre, se usa en la carátula y se puede usar el cualquier lugar del documento con el comando \authorname
\supervisor{Nombre del Director (pertenencia)} % El nombre del director, se usa en la carátula y se puede usar el cualquier lugar del documento con el comando \supname
\juradoUNO{Nombre del jurado 1 (pertenencia)} % Nombre y pertenencia del un jurado se usa en la carátula y se puede usar el cualquier lugar del documento con el comando \jur1name
\juradoDOS{Nombre del jurado 2 (pertenencia)} % Nombre y pertenencia del un jurado se usa en la carátula y se puede usar el cualquier lugar del documento con el comando \jur2name
\juradoTRES{Nombre del jurado 3 (pertenencia)} % Nombre y pertenencia del un jurado se usa en la carátula y se puede usar el cualquier lugar del documento con el comando \jur3name
\fechaINICIO{marzo de 2020}
\fechaFINAL{diciembre de 2020}

\subject{Memoria del Trabajo Final de la Carrera de Especialización en Sistemas Embebidos de la UBA} % Your subject area, this is not currently used anywhere in the template, print it elsewhere with \subjectname
\keywords{Sistemas Embebidos, CESE, FIUBA} % Keywords for your thesis, this is not currently used anywhere in the template, print it elsewhere with \keywordnames
\university{Universidad de Buenos Aires} % Your university's name and URL, this is used in the title page and abstract, print it elsewhere with \univname
\faculty{{Facultad de Ingeniería}} % Your faculty's name and URL, this is used in the title page and abstract, print it elsewhere with \facname
\department{Departamento de Electrónica} % Your department's name and URL, this is used in the title page and abstract, print it elsewhere with \deptname
\group{{Laboratorio de Sistemas Embebidos}} % Your research group's name and URL, this is used in the title page, print it elsewhere with \groupname


\hypersetup{pdftitle=\ttitle} % Set the PDF's title to your title
\hypersetup{pdfauthor=\authorname} % Set the PDF's author to your name
\hypersetup{pdfkeywords=\keywordnames} % Set the PDF's keywords to your keywords


\newcaptionname{spanish}{\acknowledgementname}{Agradecimientos}
\newcaptionname{spanish}{\authorshipname}{Declaración de Autoría}
\newcaptionname{spanish}{\abbrevname}{Glosario}
\newcaptionname{spanish}{\byname}{por}

\renewcommand{\lstlistingname}{Algoritmo}% Listing -> Algorithm
\renewcommand{\lstlistlistingname}{Índice de \lstlistingname s}% List of Listings -> List of Algorithms

\renewcommand{\listtablename}{Índice de Tablas}
\renewcommand{\tablename}{Tabla} 

\addtolength{\footnotesep}{2mm} % Espacio adicional en los footnotes

\begin{document}

\frontmatter % Use roman page numbering style (i, ii, iii, iv...) for the pre-content pages

\pagestyle{plain} % Default to the plain heading style until the thesis style is called for the body content

%----------------------------------------------------------------------------------------
%	CARÁTULA
%----------------------------------------------------------------------------------------

\begin{titlepage}
\begin{center}


\includegraphics[width=.8\textwidth]{./Figures/logoFIUBA.png}
\vspace{2cm}

\textsc{\huge{Carrera de Especialización en\\ \vspace{5px} Sistemas Embebidos}}
\vspace{.5cm} % Thesis type

\textsc{\Large Memoria del Trabajo Final}\\[1cm] % Thesis type
%\vspace{1.5cm}
{\huge \bfseries \ttitle\par}\vspace{0.4cm} % Thesis title

\vfill

\vspace{2cm}
\LARGE\textbf{Autor:\\
\authorname}\\ % Author name

\vspace{1.5cm}

\large
{Director:} \\
{\supname} % Supervisor name
 
\vspace{1cm}
Jurados:\\	
\jurunoname\\
\jurdosname\\
\jurtresname

\vspace{2cm}

\textit{Este trabajo fue realizado en las Ciudad Autónoma de Buenos Aires,\\ entre \fechaINICIOname \hspace{1px} y \fechaFINALname.}
\end{center}
\end{titlepage}


%----------------------------------------------------------------------------------------
%	RESUMEN - ABSTRACT 
%----------------------------------------------------------------------------------------

\begin{abstract}
\addchaptertocentry{\abstractname} % Add the abstract to the table of contents
%
%The Thesis Abstract is written here (and usually kept to just this page). The page is kept centered vertically so can expand into the blank space above the title too\ldots
\centering

El resumen debe escribirse en uno o dos párrafo.  Debe ser breve y conciso sin ningún elemento de formato en el texto como cursiva o negrita. Palabras que no sean del castellano deben ir en itálicas o cursiva. Tampoco se deben usar siglas ni acrónimos que no resulten obvios para un lector promedio de la memoria, ni referencias bibliográficas o notas al pie de página.  No debe faltar qué es lo que se hizo/logró, qué importancia/valor tiene el proyecto/resultado, qué va a encontrar el lector en la memoria y qué contenidos de la Especialización/Maestría se aplicaron en el proyecto.

\end{abstract}

%----------------------------------------------------------------------------------------
%	CONTENIDO DE LA MEMORIA  - AGRADECIMIENTOS
%----------------------------------------------------------------------------------------

\begin{acknowledgements}
%\addchaptertocentry{\acknowledgementname} % Descomentando esta línea se puede agregar los agradecimientos al índice
\vspace{1.5cm}

Esta sección es para agradecimientos personales y es totalmente \textbf{OPCIONAL}.  

\end{acknowledgements}

%----------------------------------------------------------------------------------------
%	LISTA DE CONTENIDOS/FIGURAS/TABLAS
%----------------------------------------------------------------------------------------
\renewcommand{\listtablename}{Índice de Tablas}

\tableofcontents % Prints the main table of contents

\listoffigures % Prints the list of figures

\listoftables % Prints the list of tables


%----------------------------------------------------------------------------------------
%	CONTENIDO DE LA MEMORIA  - DEDICATORIA
%----------------------------------------------------------------------------------------

\dedicatory{\textbf{Dedicado a... [OPCIONAL]}}  % escribir acá si se desea una dedicatoria

%----------------------------------------------------------------------------------------
%	CONTENIDO DE LA MEMORIA  - CAPÍTULOS
%----------------------------------------------------------------------------------------

\mainmatter % Begin numeric (1,2,3...) page numbering

\pagestyle{thesis} % Return the page headers back to the "thesis" style

\renewcommand{\tablename}{Tabla} 

% Incluir los capítulos como archivos separados desde la carpeta Chapters
% Descomentar las líneas a medida que se escriben los capítulos

% Chapter 1
\chapter{Introducción general} % Main chapter title

\label{Chapter1} % For referencing the chapter elsewhere, use \ref{Chapter1} 
\label{IntroGeneral}

%----------------------------------------------------------------------------------------

% Define some commands to keep the formatting separated from the content 
\newcommand{\keyword}[1]{\textbf{#1}}
\newcommand{\tabhead}[1]{\textbf{#1}}
\newcommand{\code}[1]{\texttt{#1}}
\newcommand{\file}[1]{\texttt{\bfseries#1}}
\newcommand{\option}[1]{\texttt{\itshape#1}}
\newcommand{\grados}{$^{\circ}$}

%----------------------------------------------------------------------------------------

%\section{Introducción}

%----------------------------------------------------------------------------------------
\section{Estaciones transformadoras}

Se denomina estación transformadora al conjunto de equipos electromecánicos responsables de convertir la energía eléctrica variando uno o más de sus principales parámetros, que son tensión y corriente. Esta conversión se logra a través del componente más importante del conjunto, el transformador. La finalidad de convertir la energía eléctrica, que puede ser elevando o reduciendo el nivel de tensión, es poder transmitir y distribuir esa energía hacia los receptores que pueden ser consumidores finales tales como residencias familiares o polos industriales.\\

Por convención se denomina Estación Transformadora (E.T.) cuando en el proceso se ven involucrados valores considerados de alta tensión (mayor a 66 kV) y Subestación Transformadora (S.E.T.) en el caso de tensiones menores a 66 kV \citep{AEA:1}.\\
Para la distribución hacia los consumidores finales, se utilizan las denominadas Subestaciones Transformadoras Aéreas (S.E.T.A.) que convierten la tensión disminuyendo su valor de media a baja tensión.\\

Para poder obtener energía eléctrica a la salida en óptimas condiciones de calidad y disponibilidad, resulta fundamental administrar y controlar los valores intrínsecos que componen la transmisión y recepción de la misma. Esto se logra a través de instrumentos de medición de tensión y corriente, tanto a la entrada (alta tensión) como a la salida (media tensión) de la conversión. Por otro lado, con el fin de mantener y preservar los equipos electromecánicos se consideran de gran importancia otros valores físicos como temperatura y humedad.\\
\section{Medidores de energía}
Llegado el momento de entregar la energía al usuario final, es indispensable cuantificarla para luego comercializarla. Las distribuidoras del servicio utilizan medidores de energía electromecánicos o electrónicos, que registran en todo momento la energía acumulada que fluye por el mismo.\\
La medición de corriente puede ser directa, vinculando los conductores de alimentación de la carga directamente al medidor o indirecta. Una medición indirecta consiste en reducir los valores de corriente de carga a través de transformadores de corriente (TI) y vincular sus secundarios al medidor \ref{fig:medicioncorrienteconti}. Este último método se emplea en casos donde la corriente calculada supera el valor permitido por el medidor de energía, por lo que es necesario multiplicar el valor de la lectura por un coeficiente correspondiente a la relación de transformación del TI.\\

% TODO: \usepackage{graphicx} required
\begin{figure}
	\centering
	\includegraphics[width=0.5\linewidth]{Figures/Medicion_corriente_con_TI}
	\caption[]{Medición indirecta de corriente empleando transformadores de corriente (TI).\protect\footnotemark}
	\label{fig:medicioncorrienteconti}
\end{figure}
\footnotetext{Imagen tomada por el autor}

% TODO: \usepackage{graphicx} required
\begin{figure}
	\centering
	\includegraphics[width=0.5\linewidth]{Figures/medidor_digital_con_complemento_gsm}
	\caption{Medidor de energía digital con complemento para telemedición mediante GSM. Imagen tomada de \citep{MYEEL}}
	\label{fig:medidordigitalconcomplementogsm}
\end{figure}
En la actualidad algunas prestadoras del servicio eléctrico han adoptado estrategias de medición inteligente similares a la presentada en la figura \ref{fig:medidordigitalconcomplementogsm}. En este esquema los equipos de medición se reportan a centros de operación a través de una red de comunicaciones móvil, como por ejemplo GSM.\\
El concepto de telemedición aporta además de lo comercial, valiosa información técnica, ya que los centros de operaciones conocen en todo momento el estado del medidor con la posibilidad de detectar fallas o la interrupción del servicio eléctrico.\\  

\section{Estado del arte y problemática identificada}
En Sudamérica, gran parte de las empresas distibuidoras de energía eléctrica y sus tercerizadas, basan parte de sus operaciones en el contacto directo con los usuarios finales mediante reclamos para informarse acerca de interrupciones en el servicio de distribución de energía eléctrica. Una vez recibido un reclamo, la prestadora de servicios envía al grupo de operaciones especializado a recorrer el área circundante al cliente y tratar de determinar el motivo de la interrupción del servicio.\\

% TODO: \usepackage{graphicx} required
\begin{figure}[h!]
	\centering
	\includegraphics[width=0.7\linewidth]{Figures/NH_aereo_bt}
	\caption{Fusible seccionador aéreo tipo NH usualmente utilizado en lineas de distribución de baja tensión}
	\label{fig:nh_aereo_bt}
\end{figure}
Un hecho común en el nordeste Argentino y particularmente en la provincia de Misiones es la destrucción de fusibles aéreos como el presentado en la figura \ref{fig:nh_aereo_bt}. Estos fusibles conectados inmediatamente a la salida de baja tensión y en serie con las líneas de distribución, cumplen la función de protección por sobrecorriente debido a picos de consumo o cortocircuitos causados por desastres naturales como el de la figura \ref{fig:arbolcaidolineabt}. Los fusibles involucrados actúan de manera correcta autodestruyendose e interrumpiendo el paso de corriente.\\
Este esquema presentado, resulta aún precario y no efectivo en cuanto a la rapidez para determinar la localización geográfica donde se ha generado una falla, lo que resulta en una inferior calidad de servicio prestado al cliente.\\
\begin{figure}[h!]
	\centering
	\includegraphics[width=0.7\linewidth]{Figures/arbol_caido_linea_bt}
	\caption{Un árbol caído sobre las líneas de distribución aéreas de baja tensión luego de una breve tormenta en la ciudad de Posadas, Misiones. Imagen tomada de \citep{Noticia_MNES}}
	\label{fig:arbolcaidolineabt}
\end{figure}

Cabe mencionar que la mayoría de las redes de distribución de baja tensión en 380/220V no poseen la capacidad de brindar algún otro servicio agregado. Algunas redes de media (33 kV) y alta tensión (132 kV) las cuales sin embargo, pueden albergar un conjunto de pelos de fibra óptica (OPGW) los cuales pueden ser utilizados para brindar servicios a terceras partes como empresas de telefonía o bien para monitoreo de la red propiamente dicha. Sin embargo, estas OPGW demuestran cierta vulnerabilidad frente a condiciones climáticas extremas tales como descargas atmosféricas y por lo tanto tienen un alto costo de mantenimiento \citep{ARTICLE:1}.\\
\citep{ARTICLE:2} y \citep{ARTICLE:3} comparten una arquitectura de 3 capas: física, red y aplicación para sistemas de smart grid. 
A partir del entorno donde residirá la aplicación y su objetivo final, surgen diferentes estrategias de control. Así también, la selección de sensores de diferente tipo tales como meteorológicos e infraestructura \citep{ARTICLE:3} o bien de cargas eléctricas que residen dentro de un entorno controlado haciendo uso de redes de diferente tipo \citep{ARTICLE:2}.\\
Las tecnologías emergentes propias de IoT tales como las redes de comunicación de baja potencia y largo alcance LPWAN \citep{rfc8376}, y las redes tipo malla se analizan y comparan en \citep{ARTICLE:4} como las tecnologías disponibles y viables para dotar de una infraestructura de comunicaciones a las redes de distribución metropolitanas. En \citep{ARTICLE:5} y \citep{Hua} se presentan sistemas de medición de temperatura autónomos utilizando transductores termoeléctricos y electromagnéticos para conversión de energía térmica o electromagnética en energía eléctrica para alimentar la electrónica involucrada y cargar un acumulador.

%----------------------------------------------------------------------------------------

\section{Objetivos y alcance}
\subsection{Objetivo general}
Desarrollar un sistema capaz de determinar valores eficaces de corriente alterna en sistemas metropolitanos de distribución de energía eléctrica en baja tensión y reportar estados a un centro de operaciones a través de una red LoRaWAN de acceso público.\\
\subsection{Objetivos específicos}
\begin{itemize}
	\item Evaluar el uso de un supercapacitor como reemplazo de una batería convencional.\\
	\item Desarrollar una electrónica de ultra bajo consumo para maximizar la autonomía de operación del supercapacitor.
\end{itemize}
\subsection{Alcances}
En el presente proyecto se desarrollan los siguientes temas:
\begin{itemize}
	\item Circuito de conversión de energía  basado en rectificadores de alta eficiencia.
	\item Acumulador de energía basado en supercapacitores
	\item Patrón de firmware implementado en el microcontrolador para optimizar el uso de energía del acumulador.
	\item Medición de valor RMS de corriente mediante transformador de corriente.
	\item Tecnología LoRaWAN.
	\item Recuperación, almacenamiento y presentación de datos generados por los nodos finales.
\end{itemize}
Si bien el proyecto es parte de un plan de creación de una PyME del autor, no es parte del alcance ni se cubren en este documento las etapa de lanzamiento de producto ni creación de la empresa.

\chapter{Introducción específica} % Main chapter title
\label{Chapter2}
%----------------------------------------------------------------------------------------
%	SECTION 1
%----------------------------------------------------------------------------------------
En este capítulo se presentan los requerimientos acordados con el cliente y los recursos de hardware y software utilizados para el desarrollo del trabajo. Se describen en las partes implementadas del HW, los servicios integrados de backend (BES) y solamente algunos aspectos relevantes del firmware que interactúa con el HW.\\
\section{Diagrama de bloques general del sistema}
El diagrama de bloques del HW a instalar \textit{in situ} es presentado en la figura \ref{fig:diagramadebloquesdelhw} y consta de cuatro bloques:
\begin{figure}[h!]
	\centering
	\includegraphics[width=1.0\linewidth]{Figures/diagrama_de_bloques_del_HW}
	\caption{Diagrama de bloques del HW para el nodo a instalar \textit{in situ}.}
	\label{fig:diagramadebloquesdelhw}
\end{figure}
\begin{enumerate}
	\item Circuito de selección de modo: un relay y su circuito de mando controlarán a que etapa del nodo se conectarán los terminales del transformador de corriente.
	\item Etapa de rectificación, acumulación de energía y elevación de tensión: compuesta por rectificador de onda completa, una etapa de filtrado y acumulación, y un circuito elevador de tensión.
	\item Etapa de medición de valor RMS de corriente: un chip dedicado toma la señal de tensión generada en bornes del resistor shunt y calcula el valor RMS. A su salida entrega un valor proporcional de tensión DC.
	\item Microcontrolador: ejecuta la lógica de negocios que rige el comportamiento del nodo, digitaliza mediciones y transmite datos a la red LoRaWAN.
\end{enumerate}
Por otro lado, el sistema también implicó el desarrollo y puesta en funcionamiento de un conjunto de servicios de backend (BES) propios del proyecto que cumplen las funciones de:
\begin{itemize}
	\item Recuperación de datos de la red LoRaWAN.
	\item Almacenamiento en una base de datos.
	\item Presentación de los datos al usuario final mediante una interfaz gráfica de usuario.
\end{itemize}
% TODO: \usepackage{graphicx} required

El requisito \ref{requerimiento_LORAWAN} impuso el uso de una red LoRaWAN como protocolo principal para la transmisión de datos generados por los nodos. Para cumplirlo se adoptó la arquitectura presentada en la figura \ref{fig:diagramadebloquesdebes}.\\
% TODO: \usepackage{graphicx} required
\begin{figure}[h!]
	\centering
	\includegraphics[width=0.8\linewidth]{Figures/diagrama_de_bloques_de_BES}
	\caption{Diagrama de bloques del FW implementado en el MC y su interacción con la red LoRaWAN y los BES privados del sistema.}
	\label{fig:diagramadebloquesdebes}
\end{figure}\\
Las mediciones son tomadas por el HW y transmitidas hacia la red LoRaWAN para luego interactuar con los BES privados que se encargan de recuperar, almacenar y presentar los datos al usuario final.\\

\section{Requerimientos acordados con el cliente}
\label{sec:requerimientos}
\begin{enumerate}
	\item Grupo de requerimientos asociados con hardware:
	\begin{enumerate}
		\item El dispositivo deberá ser de tipo \textit{plug and play}.
		\item El circuito impreso no deberá ocupar un volumen mayor a 10x10x5 cm.
		\item Basarse en un microcontrolador ESP32 y disponer de:
		\begin{enumerate}%[label*=\arabic*.]
			\item 4 entradas analógicas.
			\item 3 salidas digitales.
			\item Unidad UART.
			\item Integrar un módulo de comunicaciones LoRa.
		\end{enumerate}
		\item Deberá tener al menos 12 horas de autonomía de funcionamiento.
		\item Bajo consumo en modo ocioso: el consumo del hardware en total, no deberá superar los 5 mA cuando no está midiendo ni transmitiendo.
		\item El circuito elevador de tensión DC-DC deberá:
		\begin{enumerate}%[label*=\arabic*.]
			\item Funcionar con tensiones menores a 2 V en la entrada.
			\item Otorgar 5 V a la salida.
			\item Ser capaz de otorgar 300 mA a la salida.
		\end{enumerate}
		\item El transformador de corriente (TI) debe:
		\begin{enumerate}%[label*=\arabic*.]
			\item Ser de tipo núcleo partido.
			\item Admitir 100 A de corriente en el circuito primario y un máximo 5 Amperes en el circuito secundario.
		\end{enumerate}
		\item \label{req_relay} El relay encargado de cambiar el modo de operación debe:
		\begin{enumerate}%[label*=\arabic*.]
			\item Ser de tipo doble inversor sin retención.
			\item Su bobina debe poder energizarse con 5 V o menos.
			\item Soportar al menos 5 A de corriente por los contactos.
		\end{enumerate} 
		\item Debe funcionar de manera independiente a la frecuencia de operación de la red 50/60 Hz.
		\item Debe funcionar de manera independiente a la tensión de fase del sistema de distribución 110/220 V.
	\end{enumerate}
	\item Grupo de requerimientos asociados con el firmware:
	\begin{enumerate}
		\item Debe manejar un módulo de comunicación LoRa y protocolo LoRaWAN.
		\item Deberá tener un porcentaje de cobertura de tests unitarios del 60\% como mínimo.
		\item Antes configurarse en modo ocioso, debe desenergizar la etapa de medición de corriente y el módulo de comunicaciones con el objeto de ahorrar energía.
	\end{enumerate}
	
	\item Grupo de requerimientos asociados con los servicios de backend:
	\label{requerimientos_backend}
	\begin{enumerate}
		\item Todos los servicios deben poder correr en una Raspberry Pi 3.
		\item El software de los BES se desarrollará en lenguaje Python.
		\item Recuperar los datos de la red LoRaWAN.\label{requerimiento_LORAWAN}
		\item Almacenar los datos en una tabla de MySQL.
		\item Interfaz gr\'{a}fica de usuario basada en Grafana.
	\end{enumerate}
	
	\item Grupo de requerimientos asociados con ensayos de integración y \textit{end-to-end}:
	\begin{enumerate}
		\item El banco de ensayos de hardware debe contar con una carga fantasma de al menos 10 Amperes y permitir realizar interrupciones de corriente de manera programada mediante una computadora adicional tipo Raspberry Pi o de manera manual.
		\item Los BES deben estar operativos al momento de realizar los ensayos.
		\item Se debe contar con un gateway de acceso a una red LoRaWAN como por ejemplo \textit{The Things Network}.
	\end{enumerate}
\end{enumerate}


\section{Detalle del hardware}
\subsection{Transformador de corriente}
Un transformador de corriente o intensidad (TI) es un dispositivo de medición utilizado para producir en su devanado secundario una corriente diferente y proporcional a la que circula por su devanado primario.\\
El principio de operación de un TI no es diferente al de un transformador de potencia convencional. A diferencia de uno de potencia, el devanado primario puede ser de una sola vuelta sobre un núcleo ferromagnético como se ve en la figura \ref{fig:dibujomedicionti}. El devanado secundario suele tener un número mayor de vueltas alrededor del núcleo que depende de que tanto se debe reducir la corriente.\\
Muchos TI tienen una relación estándar de 5 Amperes en el secundario, por ejemplo un TI 200/5 significa que cuando por el primario fluyen 200 amperes en el secundario solo fluyen 5 A. Es decir, el TI tiene una relación de transformación de corriente N de 40 veces.\\
% TODO: \usepackage{graphicx} required
\begin{figure}[h!]
	\centering
	\includegraphics[width=0.8\linewidth]{Figures/dibujo_medicion_TI}
	\caption{Circuito de medición indirecta de corriente mediante un TI \citep{hioki}.}
	\label{fig:dibujomedicionti}
\end{figure}\\
Mediante esta técnica, pequeños instrumentos pueden monitorear grandes valores de corriente manteniendo una distancia segura de las líneas de alta tensión.

\subsection{Circuito de selección}
A partir del lineamiento de que el TI debe estar conectado por defecto a la entrada del rectificador y al resistor shunt cuando se energiza la bobina del relay, el número y la disposición de los contactos resulta un factor relevante al momento de elegir la mejor opción. La variante comercial que cumple con el requisito \ref{req_relay} es la producida por la firma Hongfa modelo HF115F/005-2ZS4A presentada en la figura \ref{fig:relay}.
\begin{figure}[h!]
    \centering
	\begin{subfigure}[b]{0.4\textwidth}
		\centering
		\includegraphics[width=.7\textwidth]{./Figures/relay_pinout}
		\caption{}
		\label{fig:relay_pinout}
	\end{subfigure}
    \centering
	\begin{subfigure}[b]{0.4\textwidth}
		\centering
		\includegraphics[width=.7\textwidth]{./Figures/relay_encapsulado}
		\caption{}
		\label{fig:relay_encapsulado}
	\end{subfigure}
	\caption{Pinout del relay HF115F/005-2ZS4A (izquierda) y su encapsulado (derecha). Imágenes tomadas de \citep{datasheet_relay}.}
	\label{fig:relay}
\end{figure}

\subsection{Conversión de energía}
Para obtener una tensión continua a partir de una alterna generada por el TI, es necesario implementar un puente rectificador de onda completa.\\
En la actualidad la mayoría de los circuitos rectificadores de onda completa se basan en diodos de silicio de bajo costo. Sin embargo, un diodo de silicio posee una caída de tensión típica de 0,7 V. Esta caída de tensión se traduce en pérdidas por efecto Joule, lo que resulta relevante en dispositivos donde la conversión, acumulación y gestión de energía es crítica. Por lo tanto, se desea maximizar la transferencia de tensión y potencia entre entrada y salida del puente rectificador.\\
Yilmaz \citep{Yilmaz} analiza técnicas de rectificación de onda completa con diferentes tipos de diodos, como así también un arreglo de transistores MOSFET pasivo y activo. Las caídas de tensión simuladas entre la entrada y salida entre un puente rectificador de diodos de silicio y uno pasivo basado en MOSFETs se comparan en la figura \ref{fig:comparacion_diodos_vs_MOSFET}.\\

\begin{figure}[h!]
	\begin{subfigure}{.5\textwidth}
		\centering
		% include first image
		\includegraphics[width=.8\linewidth]{Figures/YILMAZ_silicon_diode_rectifier}  
		\caption{Rectificador basado en diodos de silicio.}
		\label{fig:rect_diodos}
	\end{subfigure}
	\begin{subfigure}{.5\textwidth}
		\centering
		% include second image
		\includegraphics[width=.8\linewidth]{Figures/onda_silicon_rectifier}  
		\caption{Caída de tensión generada por el rectificador de la figura \ref{fig:rect_diodos}.}
		\label{fig:onda_rectificador_diodos}
	\end{subfigure}
	\newline
	\begin{subfigure}{.5\textwidth}
		\centering
		% include third image
		\includegraphics[width=.8\linewidth]{Figures/YILMAZ_passive_MOSFET_rectifier}  
		\caption{Rectificador basado en transistores MOSFET.}
		\label{fig:rect_MOSFET}
	\end{subfigure}
	\begin{subfigure}{.5\textwidth}
		\centering
		% include fourth image
		\includegraphics[width=.8\linewidth]{Figures/onda_passive_mosfet_rectifier}  
		\caption{Caída de tensión generada por el rectificador de la figura \ref{fig:rect_MOSFET}.}
		\label{fig:onda_rectificador_MOSFET}
	\end{subfigure}
	\caption{Simulación de rectificadores basados en diodos y MOSFET. Imágenes tomadas de \citep{Yilmaz}.}
	\label{fig:comparacion_diodos_vs_MOSFET}
\end{figure}
Al comparar las simulaciones expuestas en las figuras \ref{fig:onda_rectificador_diodos} y \ref{fig:onda_rectificador_MOSFET} se aprecia que la caída de tensión generada por el rectificador basado en MOSFET al entrar en conducción es menor que uno hecho con diodos y por lo tanto también la potencia disipada en forma de calor.\\

\subsection{Supercapacitor como acumulador de energía}
La decisión de optar por un banco de supercapacitores (SC) como reemplazo total de una batería, se basa principalmente en el entorno donde operará el nodo HW. Las diferencias entre una batería y un SC de interés para este proyecto, son plasmadas en la tabla \ref{tab:batteria_vs_supercap}.\\
Los datos meteorológicos de la provincia de Misiones presentados en \citep{historico_temperaturas}, muestran temperaturas por encima de 30 \textdegree C durante el período de septiembre a marzo. A diferencia de un SC que posee un rango de temperaturas de operación desde los -40 \textdegree C hasta 70 \textdegree C \citep{datasheet_supercap}, condiciones por encima de 35 \textdegree C resultan nocivas para una batería y generan el deterioro prematuro de sus componentes\citep{MA2018653}.\\
%\begin{verbatim}
\begin{table}[h]
	\centering
	\caption{Comparativa entre una batería y un supercapacitor para este proyecto.}
	\begin{tabular}{lcc} 
		\hline
		\multicolumn{1}{c}{}                                                  & Batería         & Supercapacitor                                                                         \\ 
		\hline
		\begin{tabular}[c]{@{}l@{}}Densidad de \\energía (Wh/Kg)\end{tabular} & 265             & 3,9                                                                                    \\
		\begin{tabular}[c]{@{}l@{}}Rango de \\temperatura (\textcelsius)\end{tabular}    & 15 a 35         & -40 a 70                                                                               \\
		Gestión de carga                                                      & V o I constante & \begin{tabular}[c]{@{}c@{}}Determinado por un\\circuito RC serie \citep{ceraolo2014fundamentals}\end{tabular}  \\
		\hline
	\end{tabular}
	\label{tab:batteria_vs_supercap}
\end{table}\\
%\end{verbatim}
Es importante remarcar que la densidad de energía que pueden almacenar también es diferente, una batería tiene una densidad de energía 60 veces mayor que un SC. Sin embargo, para esta aplicación puntual no representó un factor importante a la hora de elegir el acumulador.\\
Por último, el ciclo de carga es más complejo en el caso de una batería. Las etapas de su curva de carga deben ser respetadas según sean a corriente o tensión constante. Esto trae acarreado implementar una electrónica adicional encargada de gestionar estos 2 parámetros. En un capacitor, la curva de carga está definida por un circuito RC serie \citep{ceraolo2014fundamentals}.\\ 

\subsection{Elevación de tensión mediante un conversor DC/DC }
Para proveer al microcontrolador y al resto de la electrónica asociada una tensión DC fija y constante, se optó por emplear un módulo comercial DC/DC ya existente en el mercado que se presenta en la figura \ref{fig:dcdcboost}.\\
% TODO: \usepackage{graphicx} required
\begin{figure}[h!]
	\centering
	\includegraphics[width=0.4\linewidth]{Figures/dcdc_boost}
	\caption{Módulo comercial DC/DC en topología boost utilizado para alimentar la electrónica.}
	\label{fig:dcdcboost}
\end{figure}\\
Su topología interna es \textit{boost} o elevador de tensión. En el HW del sistema, cumple la función de llevar la tensión variable del SC conectado a su entrada a una fija de 5 V. 
A su entrada admite tensiones variables desde 0,9 V hasta 5 V y puede otorgar hasta 500 miliamperes de corriente a la salida.

\subsection{Microcontrolador y firmware}
El MC es el ente encargado de ejecutar la lógica de negocios acorde a la tarea que debe cumplir el HW. En el mercado existe una amplia gama de fabricantes de placas de desarrollo que permiten acelerar la etapa de prototipado y validación de diseño.\\
La placa de desarrollo elegida para el prototipo fue la LoPy 4 producida por la firma Pycom que se presenta en la figura \ref{fig:lopy4} . En su interior alberga un ESP32, 8 Megabytes de memoria flash, transceptores de radio LoRa y 802.11 y un regulador de tensión.\\
\begin{figure}[h]
	\centering
	\includegraphics[width=0.95\linewidth]{Figures/lopy4}
	\caption{Placa de desarrollo LoPy 4 \citep{lopy4}.}
	\label{fig:lopy4}
\end{figure}\\
El lenguaje de programación de la LoPy4 es Micropython \citep{micropy}, un lenguaje de alto nivel lanzado por primera vez en el año 2014. Desde su lanzamiento y hasta la fecha de desarrollo de este trabajo, se presenta como una variante de Python atractiva para prototipar FW sobre microcontroladores utilizando el paradigma de programación orientada a objetos.\\


\section{Detalle del software}
\subsection{Red LoRaWAN}
LoRaWAN (Long Range Wide Area Network) es un protocolo de control de acceso al medio (MAC - \textit{Medium Acces Control}) definido por \textit{LoRa Aliance} \citep{lora_alliance}. Tiene por objeto permitir la conexión de nodos de baja potencia, generalmente alimentados a batería y sin capacidad de manejo de protocolos de enrutamiento, con aplicaciones finales conectadas a Internet mediante una conexión inalámbrica de largo alcance utilizando modulación LoRa.\\
Las puertas de enlace (GW - \textit{gateways}), están conectadas al servidor central mediante conexiones IP (Internet Protocol) estándar, cumpliendo la función de puente, es decir, convierten los paquetes de radiofrecuencia (RF) en paquetes IP y viceversa \ref{fig:arqlorawan}.\\
% TODO: \usepackage{graphicx} required
\begin{figure}[h]
	\centering
	\includegraphics[width=1.05\linewidth]{Figures/arq_lorawan_2}
	\caption{Arquitectura de una red LoRaWAN y sus posibles integraciones con terceras partes \citep{lora_alliance}.}
	\label{fig:arqlorawan}
\end{figure}\\
El protocolo LoRaWAN no es un protocolo IP, por lo tanto, los paquetes del mismo necesitan de un enrutamiento y procesamiento correspondiente antes de ser entregados a la aplicación final.\\
La figura \ref{fig:sensor-gw-architecture-lora} presenta la topología tipo ``estrella de estrellas`` que adopta una red LoRaWAN. En ella, los \textit{gateways} retransmiten los mensajes recibidos de los nodos finales hacia un servidor central. La comunicación inalámbrica entre nodos y gateways aprovecha las características propias de la capa física, lo que permite establecer enlaces desde un nodo hacia uno o más de estos gateways.\\
% TODO: \usepackage{graphicx} required
\begin{figure}[h]
	\centering
	\includegraphics[width=0.9\linewidth]{Figures/sensor-gw-architecture-lora}
	\caption{Topología de una red LoRaWAN y la interacción entre los diferentes miembros \citep{lora_alliance}.}
	\label{fig:sensor-gw-architecture-lora}
\end{figure}

\subsection{Motor de base de datos}
Una base de datos almacena la información de manera ordenada en tablas o estructuras de datos. Las distintas aplicaciones pueden ejecutar consultas para solicitar la parte de esos datos que necesiten en ese momento.\\
Las bases de datos más utilizadas para aplicaciones \textit{web} son las basadas en un modelo relacional, también conocidas como bases de datos SQL (Structured Query Language). Sin embargo, en años recientes se han popularizado las llamadas bases de datos no relacionales o NoSQL. Estas últimas están orientadas a documentos y almacenan información de un mismo tipo en la forma de clave-valor.\\
Por su naturaleza, este proyecto parece indicado para utilizar una base de datos no relacional. Ya que solo almacena registros de un tipo y no necesita crear relaciones complejas entre distintas tablas de datos.\\
En principio se pensó en utilizar \textit{Elasticsearch} \citep{elasticsearch}, un servidor de búsqueda de texto basado en documentos JSON. Se trata de una herramienta open source que se integra con Grafana, la herramienta para el desarrollo de la interfaz gráfica \citep{grafana}.\\
Sin embargo, en las primeras pruebas llevadas a cabo se pudo apreciar que el consumo de memoria y procesamiento eran muy elevados si se lo implementaba en un hardware de bajas prestaciones como las \textit{Raspberry Pi} \citep{raspi}. Finalmente, se optó por utilizar MariaDB \citep{mariadb}, una base de datos del tipo SQL, también open source, con menos demanda de recursos y muy popular en aplicaciones \textit{web}.\\

\subsection{Interfaz gráfica de usuario}
Una vez recuperados los datos de la red LoRaWAN y almacenados en la base de datos, la interfaz de usuario del sistema presenta al usuario final del centro de operaciones los datos recolectados por cada nodo.\\
Una interfaz gráfica como la que se visualiza en la figura \ref{fig:guirequeridaporelcliente}, se encarga de presentar al usuario final los últimos datos adquiridos por cada nodo. De esta manera, se puede identificar de manera simple mediante un punto verde o rojo sobre el mapa si la línea monitoreada presenta un problema.
% TODO: \usepackage{graphicx} required
\begin{figure}[h!]
	\centering
	\includegraphics[width=0.9\linewidth]{Figures/GUI_requerida_por_el_cliente}
	\caption{Interfaz gráfica que muestra los datos capturados de cada nodo del sistema.}
	\label{fig:guirequeridaporelcliente}
\end{figure}\\
Para la presentación de la información se optó por Grafana, una aplicación \textit{web} de código abierto para el análisis y visualización de datos \citep{grafana}.\\
Para mejorar aun más la experiencia del usuario se complementa con el plug-in \textit{Worldmap Panel}, que permite mostrar información temporal sobre un mapa. Esta información se presenta como círculos en las coordenadas donde se encuentra ubicado el nodo que genera la información.\\
 
\chapter{Diseño e implementación} % Main chapter title

\label{Chapter3} % Change X to a consecutive number; for referencing this chapter elsewhere, use \ref{ChapterX}

\definecolor{mygreen}{rgb}{0,0.6,0}
\definecolor{mygray}{rgb}{0.5,0.5,0.5}
\definecolor{mymauve}{rgb}{0.58,0,0.82}

%%%%%%%%%%%%%%%%%%%%%%%%%%%%%%%%%%%%%%%%%%%%%%%%%%%%%%%%%%%%%%%%%%%%%%%%%%%%%
% parámetros para configurar el formato del código en los entornos lstlisting
%%%%%%%%%%%%%%%%%%%%%%%%%%%%%%%%%%%%%%%%%%%%%%%%%%%%%%%%%%%%%%%%%%%%%%%%%%%%%
\lstset{ %
  backgroundcolor=\color{white},   % choose the background color; you must add \usepackage{color} or \usepackage{xcolor}
  basicstyle=\footnotesize,        % the size of the fonts that are used for the code
  breakatwhitespace=false,         % sets if automatic breaks should only happen at whitespace
  breaklines=true,                 % sets automatic line breaking
  captionpos=b,                    % sets the caption-position to bottom
  commentstyle=\color{mygreen},    % comment style
  deletekeywords={...},            % if you want to delete keywords from the given language
  %escapeinside={\%*}{*)},          % if you want to add LaTeX within your code
  %extendedchars=true,              % lets you use non-ASCII characters; for 8-bits encodings only, does not work with UTF-8
  %frame=single,	                % adds a frame around the code
  keepspaces=true,                 % keeps spaces in text, useful for keeping indentation of code (possibly needs columns=flexible)
  keywordstyle=\color{blue},       % keyword style
  language=[ANSI]C,                % the language of the code
  %otherkeywords={*,...},           % if you want to add more keywords to the set
  numbers=left,                    % where to put the line-numbers; possible values are (none, left, right)
  numbersep=5pt,                   % how far the line-numbers are from the code
  numberstyle=\tiny\color{mygray}, % the style that is used for the line-numbers
  rulecolor=\color{black},         % if not set, the frame-color may be changed on line-breaks within not-black text (e.g. comments (green here))
  showspaces=false,                % show spaces everywhere adding particular underscores; it overrides 'showstringspaces'
  showstringspaces=false,          % underline spaces within strings only
  showtabs=false,                  % show tabs within strings adding particular underscores
  stepnumber=1,                    % the step between two line-numbers. If it's 1, each line will be numbered
  stringstyle=\color{mymauve},     % string literal style
  tabsize=2,	                   % sets default tabsize to 2 spaces
  title=\lstname,                  % show the filename of files included with \lstinputlisting; also try caption instead of title
  morecomment=[s]{/*}{*/}
}


%----------------------------------------------------------------------------------------
%	SECTION 1
%----------------------------------------------------------------------------------------
\section{Detalle del hardware por etapas}
\subsection{Circuito de selección de modo}
Considerando el caso límite donde el nivel del acumulador es bajo y el MC ordena al relay (RL) con retención cambiar su estado, la energía necesaria para excitar al RL puede generar una caída de tensión significativa en el acumulador y resultar en un circuito totalmente desenergizado sin posibilidad de volver a entrar en operación a no ser mediante la intervención humana.\\
Un caso extremo como el mencionado, implica que el HW posea un mecanismo para restablecer su operación tras largos periodos de tiempo aun después de haberse vaciado el acumulador.\\
En la figura \ref{fig:ctoselecciondemodo} se implementó el RL sin retención presentado en \ref{fig:relay} y su circuito de mando. Los terminales del TI, se encuentran conectados por defecto a la etapa de conversión y acumulacion de energía (terminales \textit{NC1} y \textit{NC2}). Una vez que la tensión en bornes del acumulador alcanza el valor de mínimo para que el conversor DC/DC entre en operación, también la electrónica sin la necesidad de una intervención externa.\\
% TODO: \usepackage{graphicx} required
\begin{figure}[h]
	\centering
	\includegraphics[width=0.7\linewidth]{Figures/cto_seleccion_de_modo}
	\caption{circuito selector de modo basado en el relay HF115-005-2ZS4}
	\label{fig:ctoselecciondemodo}
\end{figure}\\
Cada vez que se desea tomar una medición de corriente, el MC enciende la bobina del relay y conecta el TI al circuito de medición de valor RMS. Al \\

\subsection{Puente rectificador de onda completa}
Para el prototipo desarrollado en este trabajo, se replicó el circuito rectificador de onda completa presentado en la FIGURA TESIS YILMAZ. Con el objeto de reducir el espacio ocupado en el PCB, se optó por utilizar un chip DMHC3025 en encapsulado SOIC-8 [REFERENCIA A DATASHEET].\\
El encapsulado de la \ref{fig:esquematicointernorctificadormosfet} alberga un puente H formado dos transistores MOSFET tipo P y dos tipo N. Las características eléctricas de relevancia para esta aplicacion son presentadas en la tabla \ref{tabla_transistores_rectificador}.
\begin{table}[h]
	\centering
	\caption{Características eléctricas de los transistores que forman parte del rectificador de onda completa.}
	\begin{tabular}{cc} 
		\hline
		\textbf{\textbf{Parámetro}} & \begin{tabular}[c]{@{}c@{}}\textbf{}\\\textbf{\textbf{Valor máximo}}\end{tabular}  \\ 
		\hline
		Rds                         & 25 miliohms                                                                        \\
		Id                          & 6 A                                                                                \\
		Vdss                        & 30 V                                                                               \\
		\hline
	\end{tabular}\label{tabla_transistores_rectificador}
\end{table}

% TODO: \usepackage{graphicx} required
\begin{figure}[h]
	\centering
	\includegraphics[width=0.7\linewidth]{Figures/esquematico_interno_rctificador_mosfet}
	\caption{Pinout y esquematico interno del DMHC3025 [referencia al datasheet].}
	\label{fig:esquematicointernorctificadormosfet}
\end{figure}
El circuito final implementado encargado de la recitificacion y filtrado de la tension DC a la salida se presenta en la figura \ref{fig:ctorectificacionfiltrado}. Dos diodos zener \textit{D1} y \textit{D2} en antiparalelo protegen el a los transistores  de picos de sobretensión recortando la onda a 27 V para cada semiciclo.\\
% TODO: \usepackage{graphicx} required
\begin{figure}
	\centering
	\includegraphics[width=0.7\linewidth]{Figures/cto_rectificacion_filtrado}
	\caption{Etapa de rectificacion, filtrado y acumulacion implementada.}
	\label{fig:ctorectificacionfiltrado}
\end{figure}\\
A su salida, un conjunto de capacitores cerámicos \textit{C5}, \textit{C6} y \textit{C7} filtran ruido de alta frecuencia para luego acumular toda la energía en el SC conectado a J5.\\

 
\section{Firmware implementado}

\section{Servicios de Backend}

\section{Integraci\'{o}n de la red LoRaWAN}

\section{Base de Datos}

\section{GUI basada en Grafana}


% Chapter Template

\chapter{Ensayos y resultados} % Main chapter title

\label{Chapter4} % Change X to a consecutive number; for referencing this chapter elsewhere, use \ref{ChapterX}

%----------------------------------------------------------------------------------------
%	SECTION 1
%----------------------------------------------------------------------------------------

\section{PCB desarrollado}
\label{sec:pruebasHW}

\section{Medidor de valor RMS}

\section{Circuito detector de cortes}

\section{Consumo en deep sleep}

\section{Autonom\'{i}a del supercapacitor}

\section{Ensayo end-to-end} 
% Chapter Template

\chapter{Conclusiones} % Main chapter title

\label{Chapter5} % Change X to a consecutive number; for referencing this chapter elsewhere, use \ref{ChapterX}


%----------------------------------------------------------------------------------------

%----------------------------------------------------------------------------------------
%	SECTION 1
%----------------------------------------------------------------------------------------

\section{Conclusiones generales }

La idea de esta sección es resaltar cuáles son los principales aportes del trabajo realizado y cómo se podría continuar. Debe ser especialmente breve y concisa. Es buena idea usar un listado para enumerar los logros obtenidos.

Algunas preguntas que pueden servir para completar este capítulo:

\begin{itemize}
\item ¿Cuál es el grado de cumplimiento de los requerimientos?
\item ¿Cuán fielmente se puedo seguir la planificación original (cronograma incluido)?
\item ¿Se manifestó algunos de los riesgos identificados en la planificación? ¿Fue efectivo el plan de mitigación? ¿Se debió aplicar alguna otra acción no contemplada previamente?
\item Si se debieron hacer modificaciones a lo planificado ¿Cuáles fueron las causas y los efectos?
\item ¿Qué técnicas resultaron útiles para el desarrollo del proyecto y cuáles no tanto?
\end{itemize}


%----------------------------------------------------------------------------------------
%	SECTION 2
%----------------------------------------------------------------------------------------
\section{Próximos pasos}

Acá se indica cómo se podría continuar el trabajo más adelante.
 

%----------------------------------------------------------------------------------------
%	CONTENIDO DE LA MEMORIA  - APÉNDICES
%----------------------------------------------------------------------------------------

\appendix % indicativo para indicarle a LaTeX los siguientes "capítulos" son apéndices

% Incluir los apéndices de la memoria como archivos separadas desde la carpeta Appendices
% Descomentar las líneas a medida que se escriben los apéndices

%\include{Appendices/AppendixA}
%\include{Appendices/AppendixB}
%\include{Appendices/AppendixC}

%----------------------------------------------------------------------------------------
%	BIBLIOGRAPHY
%----------------------------------------------------------------------------------------

\Urlmuskip=0mu plus 1mu\relax
\raggedright
\printbibliography[heading=bibintoc]

%----------------------------------------------------------------------------------------

\end{document}  
